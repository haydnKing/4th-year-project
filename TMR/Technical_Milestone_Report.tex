%%%%%%%%%%%%%%%%%%%%%%%%%%%%%%%%%%%%%%%%%
% Journal Article
% LaTeX Template
% Version 1.1 (25/11/12)
%
% This template has been downloaded from:
% http://www.LaTeXTemplates.com
%
% Original author:
% Frits Wenneker (http://www.howtotex.com)
%
% License:
% CC BY-NC-SA 3.0 (http://creativecommons.org/licenses/by-nc-sa/3.0/)
%
%%%%%%%%%%%%%%%%%%%%%%%%%%%%%%%%%%%%%%%%%

%----------------------------------------------------------------------------------------
%	PACKAGES AND OTHER DOCUMENT CONFIGURATIONS
%----------------------------------------------------------------------------------------

\documentclass[twoside,a4paper]{article}

\usepackage{lipsum} % Package to generate dummy text throughout this template

\usepackage{textcomp}
\usepackage[sc]{mathpazo} % Use the Palatino font
\usepackage[T1]{fontenc} % Use 8-bit encoding that has 256 glyphs
\linespread{1.05} % Line spacing - Palatino needs more space between lines
\usepackage{microtype} % Slightly tweak font spacing for aesthetics

\usepackage[hmarginratio=1:1,top=32mm,columnsep=20pt]{geometry} % Document margins
\usepackage{multicol} % Used for the two-column layout of the document
\usepackage{hyperref} % For hyperlinks in the PDF

\usepackage[hang, small,labelfont=bf,up,textfont=it,up]{caption} % Custom captions under/above floats in tables or figures
\usepackage{booktabs} % Horizontal rules in tables
\usepackage{float} % Required for tables and figures in the multi-column environment - they need to be placed in specific locations with the [H] (e.g. \begin{table}[H])

\usepackage{lettrine} % The lettrine is the first enlarged letter at the beginning of the text
\usepackage{paralist} % Used for the compactitem environment which makes bullet points with less space between them

\usepackage{abstract} % Allows abstract customization
\renewcommand{\abstractnamefont}{\normalfont\bfseries} % Set the "Abstract" text to bold
\renewcommand{\abstracttextfont}{\normalfont\small\itshape} % Set the abstract itself to small italic text

\usepackage{titlesec} % Allows customization of titles
\renewcommand\thesection{\Roman{section}}
\titleformat{\section}[block]{\large\scshape\centering}{\thesection.}{1em}{} % Change the look of the section titles

\usepackage{fancyhdr} % Headers and footers
\pagestyle{fancy} % All pages have headers and footers
\fancyhead{} % Blank out the default header
\fancyfoot{} % Blank out the default footer
\fancyhead[C]{Haydn King $\bullet$ Technical Milestone Report $\bullet$ December 2012} % Custom header text
\fancyfoot[RO,LE]{\thepage} % Custom footer text

%----------------------------------------------------------------------------------------
%	TITLE SECTION
%----------------------------------------------------------------------------------------

\title{\vspace{-15mm}\fontsize{24pt}{10pt}\selectfont\textbf{A Cross-Genome
Study of the Pentatricopeptide Repeat (PPR) Protein}} % Article title

\author{
\large
\textsc{Haydn King}\thanks{Supervised by: \textsc{Dr Jorge Gon\c{c}alves} 
(Dept. of Engineering), \textsc{Dr Jim Haseloff} 
(Dept. of Plant Sciences)}\\[2mm] % Your name
\normalsize Department of Engineering \\
\normalsize University of Cambridge \\ % Your institution
\normalsize hjk38@cam.ac.uk % Your email address
\vspace{-5mm}
}
\date{}

%----------------------------------------------------------------------------------------

\begin{document}

\maketitle % Insert title

\thispagestyle{fancy} % All pages have headers and footers

%----------------------------------------------------------------------------------------
%	ABSTRACT
%----------------------------------------------------------------------------------------

\begin{abstract}

\noindent \lipsum[1] % Dummy abstract text

\end{abstract}

%----------------------------------------------------------------------------------------
%	ARTICLE CONTENTS
%----------------------------------------------------------------------------------------

\begin{multicols}{2} % Two-column layout throughout the main article text

\section{Introduction and Motivation}

\lettrine[nindent=0em,lines=3]{S} ynthetic biology is a relatively new
engineering discipline whose goal is to apply standard engineering techniques
such as standardisation, characterisation and encapsulation of function to 
biology.
Synbio aims to use design principles to combine existing phenomena to build 
new, artificial forms of life.
The field is often confused with its spiritual predecessor, genetic 
engineering, which although similar in some respects does not design new
organisms, but tinkers with existing ones without trying to understand the
underlying principals.
For a brief introduction to those principals, see appendix~\ref{sec:mbio}.

Synbio is often referred to as programming, but with DNA instead of computer
code.
An example project which captures this idea is Tabor's bacterial edge
detector\cite{edgeDetector}.
Bacteria were programmed to produce a colourless chemical messenger in the 
absence of light and to produce a dark pigment in the presence of light and the
chemical messenger.
When a film these bacteria is exposed to a pattern of light and dark, the
messenger diffuses out from the dark regions and into the light, where it
stimulates the production of the pigment, leading to an edge detection effect.

While this and other such simple demonstration shows some of the potential 
of synbio, they lacks immediate application and are somewhat limited.
A major problem in expanding this work is the lack of targeted reporter
molecules.
In the edge-detector example, two molecular signal are produced when light is not 
present -- AHL, a cell-to-cell signalling molecule and cl, a
transcriptional repressor molecule.
Both AHL and cl affect the promoter $P_{lux-\lambda}$; while AHL stimulates
expression, cl strongly represses it.
With expression of the dark pigment being driven by $P_{lux-\lambda}$, 
both light and AHL are required to cause the pigment to be produced.

The effect of the molecules AHL and cl on $P_{lux-\lambda}$ is one of a small
but growing number of well understood control motifs.
Since reusing the same promoter/signal in the same cell is impossible due to 
cross-talk, there are simply not enough signalling modalities available to
perform more complex calculations within the cell.
Indeed, it is often the case that signalling molecules have multiple functions
within the cell such that changing the concentration of one molecule to suit
our goals may cause a seemingly unrelated are of the cells metabolism to
malfunction with undesirable consequences.

Another successful synbio project is the effort to produce artemisinin (the
most effective known anti-malarial) in a cheaper and more scalable way.
Artemisinin is found naturally in sweet wormwood, but it is slow and expensive
to extract directly from the plant and chemical synthesis is also an expensive
and laborious process.
Synthetic biologists were able to extract the metabolic pathway responsible for
the biosynthesis of artemisinic acid (a natural precursor) and insert it into 
yeast\cite{yeast}.
Artemisinin produced in this manner has yet to be approved for sale, but it is
hoped that it should be available at some point during 2013, at a considerably
lower price than any other method of production.

The major limiting factor in this project was yield.
In order to produce a useful amount of the drug, the pathway involved had to be
up-regulated, which led to a difficult balance -- too little and very little
artemisinic acid would be produced, too high and too much of the cell's
energy would be used, causing the cells to grow slowly if at all.
As well as this, growing yeast on an industrial scale relatively expensive.
It is desirable therefore search for host platforms which are better suited to
biosynthesis than yeast, in order to maximise the yield to cost ratio.

Chloroplasts are a major centre for biosynthesis in plants as they perform
photosynthesis to provide energy for the plant.
The result of an ancient symbiosis, up to 1000 of these primitive cells can be 
found within each plant cell, where they make an excellent target for synbio.
They are similar to previous synbio hosts, but with access to the more
sophisticated plant cell machinery and superb potential for biosynthesis.
The native enzyme, RuBisCO, is expressed in the chloroplasts where it makes up
up to 50\% of soluble leaf protein.
Achieving anything remotely close to this figure in a project such as the
production of artemisinin would help reduce the vast number of people who die 
of this treatable disease each year (roughly 2,000 deaths a day in 2010
\cite{malaria}).

%------------------------------------------------

\section{Project Background}

Understanding how gene expression is controlled in chloroplasts is a key step
in achieving this goal.
Unlike previous synbio targets, chloroplast genes are generally expressed
constitutively (continuously) leading to constant mRNA levels rather than 
being controlled by promoter regulation\cite{Sugita1996}.

PPR proteins are a class of signalling protein found almost exclusively in
plants\cite{Small2000}. 
They are synthesised in the nucleus and sent to organelles such as the 
chloroplast where each one binds to a specific RNA sequence.
Depending on whether the protein binds to the untranslated region within the
RNA message or over the ribosome binding site, expression is either increased 
(by preventing exonucleases from destroying the mRNA) or decreased by reducing 
ribosome activity\cite{Pfalz2009}.

PPR proteins are made up of a short targeting region followed by a series of
repeating regions which form the RNA binding site.
Often, the protein also contains a tail region which can be classified as
belonging to one of 3 different categories, 
but the function of which is hitherto unknown\cite{Lurin2004}.

Discovering the target region of a given PPR protein is a difficult problem as
nothing is known of the 3D structure of the repeat domains.
Statistical analysis of a large set of experimentally verified PPR/RNA pairs
has shown a strong link between the amino acids at position 6 in repeat domain
$n$ and position 1 in repeat domain $n+1$ (hitherto refered to position 1')
and the bound nucleotide at position $n$\cite{Barkan2012}.
Weight was added to this conjecture after a PPR protein with known binding
target was mutated such as to change its binding preference in a predictable
way\cite{Barkan2012}.

Ultimately, being able to design a PPR protein to bind to an arbitrary RNA
sequence with a pre-specified affinity would be of great use to synthetic
biology.
It would both improve our knowledge of the chloroplast and give us the
necessary tools to control expression within it and provide a convenient way to
precisely control the expression of a gene without the possibility of 
cross-talk or interference.

Unfortunately, such an undertaking is infeasible for such a short project, and
so the goals of this project are:
\begin{itemize}
  \item Find and predict the binding targets of PPR proteins from several plant 
    genomes and compare those which correspond to
    similar binding targets and discover which features of the protein are
    preserved
  \item Verify that PPR-style control can be performed in a bacterial setting
    such as in \textit{E. coli}, by designing and performing a simple test 
    with known PPR/RNA binding pairs
\end{itemize}


%------------------------------------------------

\section{Preliminary Results}

\begin{table}[H]
\caption{Example table}
\centering
\begin{tabular}{llr}
\toprule
\multicolumn{2}{c}{Name} \\
\cmidrule(r){1-2}
First name & Last Name & Grade \\
\midrule
John & Doe & $7.5$ \\
Richard & Miles & $2$ \\
\bottomrule
\end{tabular}
\end{table}

\lipsum[5] % Dummy text

\begin{equation}
\label{eq:emc}
e = mc^2
\end{equation}

\lipsum[6] % Dummy text

%------------------------------------------------

\section{Further Work}

\lipsum[7-8] % Dummy text

%----------------------------------------------------------------------------------------
%	APPENDICES
%----------------------------------------------------------------------------------------

\begin{center}
  \large\textsc{Appendices}
\end{center}

\appendix

\section{Hidden Markov Models and the HMMER Package}
\label{sec:HMMs}

A Hidden Markov Model (HMM) is a statistical model of a Markov Process where
the sequence of states is unknown but a symbol is emitted from each state. 
An HMM has a set of $N$ states, 
$\Omega = \omega_{1 \ldots N}$ and an alphabet of $M$ symbols, 
$\Psi = \psi_{1 \ldots M}$. 
The probability of emitting the symbol $\psi_i$ from state $\omega_j$ is 
defined as $\theta_{\psi_i | \omega_j}$.
Similarly, the probability of transitioning from state $\omega_j$ to state
$\omega_i$ is given by $\phi_{\omega_i | \omega_j}$.
The model results in a sequence of states $x(t) \in \Omega$ which are not
observed, and a sequence of symbols $y(t) \in \Psi$ which are observed for some
range of $t$.

HMMs have proved useful in a number of fields, but have been particularly
useful in modeling biological sequences. 
In general, bioinformaticians use a special case of the HMM called a
profile-HMM or a pHMM.
A pHMM is an HMM whose network topology is fixed, as shown in.
They contain a number of nodes, each of which contains an emission state, an
insert state and a mute delete state, which either emit a single symbol, emit
one or more symbols or emit no symbols before moving to the next node
respectively.

Profile-HMMs have numerous practical advantages over general HMMs. 
Firstly,there is a significant reduction in the number of transition states
which must calculated and stored and secondly it possible to automatically
generate a pHMM from a sequence alignment using the Expectation-Maximisation
algorithm as the topology is fixed. More information is available about HMMs
and other aspects of bioinformatics in \cite{Durbin1998}.

Many of the algorithms required to build and manipulate pHMMs are implemented
in the \textsc{hmmer}\cite{HMMERguide} package, a free and open source software 
package available from \href{http://hmmer.janelia.org/}{hmmer.janelia.org}.

\section{Molecular Biology}
\label{sec:mbio}

Molecular biology is the study of the molecular basis of biology.
While the field itself is rather broad, much of it is underpinned by what is
refered to as the central dogma of molecular biology.
This central dogma describes the flow of information within a cell and the
processes and control mechanisms which regulate this process.
Naturally, many of these processes are highly complicated and poorly
understood, but much progress has been made since the discovery of DNA in the
1960s to understand these processes.
Below is a brief introduction, aimed at the information or control engineer.

Molecules of DNA are the cell's long term storage mechanism -- recent research
estimates the half-life of DNA to be 521 years\cite{DNAhalflife}.
The first process is called \textit{translation}, where the DNA molecule is
'read' by an RNA polymerase, producing an RNA copy of a section of the DNA.
The RNA molecule is called messenger-RNA as it is a short-lived (minutes to
hours) message.
This message is read by a ribosome, a molecule which translates the mRNA into a
protein, a process refered to as \textit{translation}.
Proteins then fold into a very specific shape determined by their 
sequence, and go on to perform many important functions within the cell.
The processes of transcription and translation are typically very tightly
controlled by the cell, as this is the main way of influencing the levels of
various proteins within the cell.

DNA consists of a sequence of four different nucleotides recorded as G,A,T and C.
When DNA is transcribed to mRNA, thymine is replaced with uracil, such that the
RNA alphabet is represented as G,A,U and C.
Proteins are a sequence of amino acids, where each acid comes from an alphabet
of 20 amino acids.
Each acid is coded for by 3 base pairs of RNA, which are referred to
collectively as a codon.
Since there are $4^3$ possible codons and only 20 amino acids, the code is
over complete -- several different codons map to the same amino acid.
As well as coding for amino acids, three special codons (UAG, UAA and UGA) are
known as stop codons as they terminate the translation of the protein.

The DNA region which codes for a protein is called a gene, and is marked by a
promoter region, to which the RNA polymerase binds at the start of
transcription.
Control is often achieved by modulating the activity of the promoter, either to
enhance or hinder the binding of RNA polymerase.
In prokaryotes, the promoter region is usually a short distance upstream from
the gene or genes to be transcribed, such that the mRNA sequence contains a
short untranslated region, followed by one or more genes and then another short
untranslated region.

Ribosomes bind to the mRNA, reading the gene and creating the appropriate
protein before detaching from the mRNA.
mRNA is more fragile than DNA but is also targeted by exonucleases, a class of
enzyme which degrade the RNA molecule, preventing it from producing more
protein.
Similar process exist which degrade proteins over time, recycling their amino
acids to form new proteins.
These degradation processes mean that a gene must continue to be transcribed at
a constant rate for the concentration of its protein to remain stable.


%----------------------------------------------------------------------------------------
%	REFERENCE LIST
%----------------------------------------------------------------------------------------

\bibliographystyle{unsrt}

\bibliography{references}

%----------------------------------------------------------------------------------------

\end{multicols}

\end{document}
