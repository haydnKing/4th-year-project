
\chapter{Probability of a Binding Domain Appearing by Chance} % Main appendix title

\label{AppendixA} % For referencing this appendix elsewhere, use \ref{AppendixA}

%\markboth{\emph{Appendix A. Probability of a Binding Domain Appearing by Chance}} 

We are interested in the probability of a particular sequence $S$ of length $N$
appearing within a larger sequence of length $M$ one or more times.
Both sequences are assumed to be made up of an alphabet of size $K$, and the
larger sequence is assumed to be a series of independent, uniform draws where
each symbol has equal probability $\frac{1}{K}$.

In order to avoid some rather involved combinatorics, we calculate the expected
number of draws until $S$ appears.
Letting $a$ equal the expected number of draws until we draw the first symbol
in $S$, we see that
\begin{align*}
  a &= 1 + \frac{K-1}{K}~a \\
  \Rightarrow a &= K
\end{align*}
since we must draw at least one symbol and there is a $\frac{K-1}{K}$ chance of
us being returned to $a$.

Now let $b$ equal the number of draws until we find the first two symbols in
$S$.
The second symbol is drawn on the $(a+1)^{\mathrm{th}}$, and these is a
$\frac{K-1}{K}$ chance of us returning to $b$ so that
\begin{align*}
  b &= (a+1) + \frac{K-1}{K}~b \\
  \Rightarrow b &= K (K+1)
\end{align*}
Having seen this pattern, we can easily invoke proof by induction to show that
the expected number of draws until the entirety of $S$ is drawn is
\begin{equation*}
  \sum_{n=1}^{n=N} K^n
\end{equation*}

This, the expected number of occurrences of S in a random sequence of length $M$
is
\begin{equation*}
  \frac{M}{\sum_{n=1}^{n=N} K^n}
\end{equation*}
We can now use the Poisson distribution to calculate the probability of at
least one occurrence of S, which is given by
\begin{equation*}
  P = e^{-\frac{M}{\sum_{n=1}^{n=N} K^n}}
\end{equation*}

Since out binding site can be located on the forward or reverse strand, we are
actually interested in finding the sequence S itself, or the reverse complement
of S, which is equally likely as out probability distribution is uniform.
The probability of finding at least one instance of S or it's reverse
complement is thus given by
\begin{equation*}
  2P - P^2
\end{equation*}

