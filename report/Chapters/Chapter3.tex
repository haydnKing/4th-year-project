
\chapter{Results} 
\label{chap:results}

\lettrine{T}{he} 
algorithm developed in section \ref{sec:ppr_extraction} was applied to the
genomes of several target organisms.
PPRs were found in similar numbers in almost all of the plants tested, implying
a high level of similarity in the method and degree of PPR interaction between
the nuclear genome and the organelles in most plants.

\section{Survey of PPRs in Plants}
\label{sec:ppr_survey}

\subsection{Selected Genomes}
\label{sec:survey_genomes}

Despite recent advances, whole genome sequencing is still somewhat expensive,
time consuming and error prone and so only a small subset of plants have been
fully sequenced.
Those genomes selected as part of the survey are shown in \ref{tab:genomes} and
consist mostly of plants, since this is where most PPRs are known to reside.

One interesting exception is \emph{P. falciparum}, which is the parasite which
causes the most dangerous form of malaria in humans.
It is included here because \emph{P. falciparum} contain an apicoplast -- 
an organelle similar to the chloroplasts found in plants and thought to be the 
result of a secondary endo-symbiosis. 

Complete sequences for the selected genomes were fetched from the NCBI genbank
genomes repository.

\begin{table}
  \centering
  \begin{tabular}{l | c | p{7cm}}
    \textbf{Genome} & \textbf{Abbrv.} & \textbf{Description} \\ \hline
    \emph{Arabidopsis thaliana   }  & At & Thale cress, winter annual \\ \hline  
    \emph{Brachypodium distachyon}  & Bd & Purple False Brome, grass species\\ \hline
    \emph{Citrus sinensis        }  & Cs & Orange \\ \hline
    \emph{Eutrema parvulum       }  & Ep & Small herb\\ \hline
    \emph{Eutrema salsugineum    }  & Es & Halophyte (tolerates high salt) \\ \hline
    \emph{Glycine max            }  & Gm & Soya Bean, legume \\ \hline
    \emph{Gossypium raimondii    }  & Gr & Cotton \\ \hline
    \emph{Malus x domestica      }  & Mx & Apple \\ \hline
    \emph{Medicago truncatula    }  & Mt & Barrel Clover, legume \\ \hline
    \emph{Oryza brachyantha      }  & Ob & Grass species, distant rice relative\\ \hline
    \emph{Oryza sativa           }  & Os & Rice \\ \hline
    \emph{Ostreococcus tauri     }  & Ot & Unicellular green algae \\ \hline
    \emph{Plasmodium falciparum  }  & Pf & Malarial parasite, contains apicoplasts \\ \hline
    \emph{Solanum lycopersicum   }  & Sl & Tomato \\ \hline
    \emph{Sorghum bicolor        }  & Sb & Grass species \\ \hline
    \emph{Zea mays               }  & Zm & Maize \\ \hline
  \end{tabular}
  \caption{\textbf{Target genomes searched for PPR proteins}
    \label{tab:genomes}}
\end{table}

\subsection{Extraction Results}
\label{sec:survey_results}

The genomes listed in table \ref{tab:genomes} were searched using the algorithm
described in section \ref{sec:ppr_extraction}.
Figure \ref{fig:ppr_numbers} shows the number and type of PPRs found in each
genome of interest, ordered by number found.

It is clear that \emph{P. falciparum} contains no PPRs at all and so
although the apicoplasts  is similar to the chloroplast, an entirely
different control mechanism is at work in them.
\emph{O. tauri} contains very few PPRs, which is unsurprising as it is a
relatively simple unicellular organism.

The majority of the plants surveyed contain between 400 and 600 PPR proteins,
however, the clear exception to this is \emph{G. max}, the soya bean, 
which contains considerably more putative PPR proteins than any other of the 
surveyed plants -- 940 in total.
The reasons behind this are a mystery, however they genome is known to contain
several repeats of each protein \citep{Schmutz2010}.

Figure~\ref{fig:ppr_family_lengths} shows a histogram of the number of motifs
found in each PPR stacked by family.
We see that the most common length is around 13 motifs, but that some PPRs are
around 30 repeat regions in length.
The number of repeat regions corresponds to the size of the binding footprint
of the protein, and thus the longer proteins are more specific than the shorter
ones, as the change of a longer sequence appearing by chance is significantly
smaller.


\begin{figure}
  \begin{center}
    \begin{tikzpicture}
      \begin{axis}[
          ybar stacked,
          xtick=data,% crucial line for the xticklabels directive
          ymin=0,
          xlabel={Genome Abbreviation},
          ylabel={Number of Proteins},
          xticklabels from table={Data/ppr_families.dat}{genome},
          width=0.9\textwidth,
          height=7cm,
          legend style = {
            at={(0.5,1.05)},
            column sep = 3mm,
            anchor=south},
          legend columns = 4
        ]

        \addplot table [x expr=\coordindex,y=type_p] {Data/ppr_families.dat};
        \addplot table [x expr=\coordindex,y=type_e] {Data/ppr_families.dat};
        \addplot table [x expr=\coordindex,y=type_ep]{Data/ppr_families.dat};
        \addplot table [x expr=\coordindex,y=type_dyw]{Data/ppr_families.dat};

        \addlegendentry{P type}
        \addlegendentry{E type}
        \addlegendentry{E+ type}
        \addlegendentry{DYW type}

      \end{axis}
    \end{tikzpicture}
    \caption{
      \textbf{The number of PPR proteins found in each genome}, 
      stacked by type (as defined in figure~\ref{fig:ppr_anatomy}).
      The two non-plants, \emph{O. tauri} and \emph{P. falciparum} contain none
      or very few, whereas most of the plants surveyed contain a similar number
      of PPRs, with the exception of \emph{G. max} which contains an unusually
      large number.
      \label{fig:ppr_numbers}}
  \end{center}
\end{figure}

\begin{figure}
  \begin{center}
    \begin{tikzpicture}
      \begin{axis}[
          ybar stacked,
          ymin=0,
          xlabel={Number of Motifs},
          ylabel={Number of Proteins},
          width=0.9\textwidth,
          height=7cm,
          bar width = 1.0,
          legend style = {
            at={(0.5,1.05)},
            column sep = 3mm,
            anchor=south},
          legend columns = 4
        ]

        \addplot table [x=length,y=p] {Data/ppr_family_lengths.dat};
        \addplot table [x=length,y=e] {Data/ppr_family_lengths.dat};
        \addplot table [x=length,y=ep]{Data/ppr_family_lengths.dat};
        \addplot table [x=length,y=dyw]{Data/ppr_family_lengths.dat};

        \addlegendentry{P type}
        \addlegendentry{E type}
        \addlegendentry{E+ type}
        \addlegendentry{DYW type}

      \end{axis}
    \end{tikzpicture}
    \caption{
      \textbf{The distribution of the number of motifs found in proteins}, 
      summed over each of the organisms surveyed and stacked by type.
      PPRs are most commonly between 10 and 15 motifs long, which gives enough
      specificity to make finding a particular sequence by chance unlikely ($p
      \approx 3.4 \times 10^{-3}$ for 13 motifs) assuming only a single 
      possible binding sequence. 
      \label{fig:ppr_family_lengths}}
  \end{center}
\end{figure}

\section{Binding Locations}

A key problem which must be overcome in order to develop PPR-based technology
is understanding how the amino acid sequence of the proteins determines the RNA
sequence to which it binds.
This knowledge could be used in two ways, firstly to predict the binding
footprints of PPRs which are known to exist in order to discover their role in
the chloroplast and secondly in order to make designing PPRs with pre-specified
binding preferences a reality.
This section focusses on the former problem.

We wish do discover a mapping between the amino acid sequence of a repeat
motif in a PPR and the RNA base to which it binds.
Two such mappings have been proposed recently, the first in \citet{Barkan2012}
and the second in \citet{Yagi2013}.
These mappings are similar in many senses, they both predict a distribution
over the four bases based on the amino acids found at particular locations
within the motif.

Both the Barkan and Yagi coding schemes were implemented in python and tested
using the PSSM prediction algorithm described in section
\ref{sec:pssm_binding}.
Characterised PPRs (given in \citet{Yagi2013}) were searched against the
\emph{A. thaliana} chromosome and the scores were compared to the average 
output for an entirely random sequence of equal length.
It was found that neither scheme produced any matches with scores high enough 
to be statistically significant.
In fact, in most cases the known PPR binding site was not even in the top 10\%
of matches.
Clearly there is some way to go in developing an understanding of this
phenomenon.

There are two main problems which came to light while studying these papers.
\begin{enumerate}
  \item PPR motifs are often not very well defined -- the two papers even use
    differing conventions on where exactly the motif starts, but neither make
    mention of this fact
  \item There are few well known PPR-RNA interactions
\end{enumerate}
The first of these problems will improve with the second -- we can only improve
our prediction of where exactly motifs should be defined when we understand
more about them.

\subsubsection{Discovering More PPR-RNA Pairs}

While the number of fully sequenced nuclear genomes remains low, a large 
number of chloroplast genomes have been sequenced.
This is due to their relatively small size (120-170kb) which makes them
considerably easier and cheaper to sequence than the nuclear genome.

Chloroplast genomes are generally well conserved between different organisms --
although genes are sometimes rearranged or duplicated, most chloroplast genomes
contain roughly the same thing.
This can be used to our advantage by looking for potential homologs of our known
PPR binding sites in other chloroplasts to predict the presence of a homologous
protein in the nuclear genome.
Homologs could then be confirmed experimentally with relative ease, as we would 
known enough about the sequence of the homologs to be able to extract them from 
the nuclear genome (e.g. by PCR) and we would also already have a good idea of 
what the binding footprint is.
Having a large group of proteins whose binding footprint changes only slightly
would be a great aid in elucidating the binding scheme.

This theory was tested using the 340 chloroplast (and closely related
organelle)
genomes available from the NCBI's Organelle Genome Resource.
Directly searching other genomes for sequences similar to known binding domains
would not produce results of much significance as the size of the genome means
that similar sequences are likely to exist there by chance.
Instead, the genomes were first searched for proteins which had high protein
sequence similarity to the gene nearest to the PPR binding site in Arabidopsis
using HMMER's \emph{phmmer} program.
The overwhelming majority of the other 339 genomes were found to have proteins 
with very strong sequence similarity to those in Arabidopsis. 
The DNA sequence of these proteins was then searched for regions similar to the
original binding sites using a PSSM approach.
Since DNA to amino acid coding is a overcomplete, even if a protein is found
with an identical amino acid sequence the DNA sequence could be very different.


