
\chapter{Results and Discussion} 
\label{chap:results}

\section{Survey of PPRs in Plants}
\label{sec:ppr_survey}


\begin{figure}
  \begin{center}
    \begin{tikzpicture}
      \begin{axis}[
          ybar stacked,
          xtick=data,% crucial line for the xticklabels directive
          ymin=0,
          xlabel={Genome Abbreviation},
          ylabel={Number of Proteins},
          xticklabels from table={Data/ppr_families.dat}{genome},
          width=0.9\textwidth,
          height=7cm,
          legend style = {
            at={(0.5,1.05)},
            column sep = 3mm,
            anchor=south},
          legend columns = 4
        ]

        \addplot table [x expr=\coordindex,y=type_p] {Data/ppr_families.dat};
        \addplot table [x expr=\coordindex,y=type_e] {Data/ppr_families.dat};
        \addplot table [x expr=\coordindex,y=type_ep]{Data/ppr_families.dat};
        \addplot table [x expr=\coordindex,y=type_dyw]{Data/ppr_families.dat};

        \addlegendentry{P type}
        \addlegendentry{E type}
        \addlegendentry{E+ type}
        \addlegendentry{DYW type}

      \end{axis}
    \end{tikzpicture}
    \caption{The number of PPR proteins found in each genome
      \label{fig:ppr_numbers}}
  \end{center}
\end{figure}

\begin{figure}
  \begin{center}
    \begin{tikzpicture}
      \begin{axis}[
          ybar stacked,
          ymin=0,
          xlabel={Number of Motifs},
          ylabel={Number of Proteins},
          width=0.9\textwidth,
          height=7cm,
          bar width = 1.0,
          legend style = {
            at={(0.5,1.05)},
            column sep = 3mm,
            anchor=south},
          legend columns = 4
        ]

        \addplot table [x=length,y=p] {Data/ppr_family_lengths.dat};
        \addplot table [x=length,y=e] {Data/ppr_family_lengths.dat};
        \addplot table [x=length,y=ep]{Data/ppr_family_lengths.dat};
        \addplot table [x=length,y=dyw]{Data/ppr_family_lengths.dat};

        \addlegendentry{P type}
        \addlegendentry{E type}
        \addlegendentry{E+ type}
        \addlegendentry{DYW type}

      \end{axis}
    \end{tikzpicture}
    \caption{Histogram showing the number of motifs found in the proteins
      \label{fig:ppr_family_numbers}}
  \end{center}
\end{figure}

\begin{figure}
  \begin{center}
    \begin{tikzpicture}
      \begin{axis}[
          ybar stacked,
          xtick=data,% crucial line for the xticklabels directive
          ymin=0,
          ymax=100,
          xlabel={Genome Abbreviation},
          ylabel={Percentage or PPRs},
          xticklabels from table={Data/ppr_localization.dat}{genome},
          width=0.9\textwidth,
          height=7cm,
          legend style = {
            at={(0.5,1.05)},
            column sep = 3mm,
            anchor=south},
          legend columns = 4
        ]

        \addplot[fill=green!50!black!20]
          table [x expr=\coordindex,y=c]{Data/ppr_localization.dat};
        \addplot table [x expr=\coordindex,y=m]{Data/ppr_localization.dat};
        \addplot table [x expr=\coordindex,y=s]{Data/ppr_localization.dat};
        \addplot table [x expr=\coordindex,y=other]{Data/ppr_localization.dat};

        \addlegendentry{Chloroplast}
        \addlegendentry{Mitochondria}
        \addlegendentry{Secretory Pathway}
        \addlegendentry{Other}

      \end{axis}
    \end{tikzpicture}
    \caption{The percentage of PPRs localized to different areas of the cell
      \label{fig:ppr_localization}}
  \end{center}
\end{figure}

A discussion of the results of the PPR survey, the number of PPRs in each plant
and their connection.

\section{Prediction of Binding Locations}

A discussion of any extra homology found between PPRs with similar binding
preferences.


