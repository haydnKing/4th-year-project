
\chapter{Simulating PPR Activity} 
\label{chap:simulation}

\lettrine{S}{ynthetic} biologist often draw on analogies with electrical
engineering when designing new systems.
Individual components are assembled to form biological circuits, where the rate
of expression is seen as analogous to the current flowing through the device.

If we imagine that we were able to solve the issues discussed in the previous
chapter and that PPRs could be designed to bind to arbitrary RNA sequences,
what kind of components could we build for our biological circuits?

In this chapter, a generalised ODE model of a network of interacting PPRs is
developed and used to show that PPRs could potentially be used to perform basic
logic operations at the translational level.

\section{Simulation Model}
\label{sec:sim_model}

\begin{figure}
  \centering
  \begin{tikzpicture}[auto,
      converts/.style={->, >=triangle 60},
      produces/.style={->, >=triangle 60, dashed},
    reversible/.style={->, >=triangle 60, bend left=20}]
    \node (D) {$\mathrm{DNA}_i$};
    \node (R) [below=2cm of D] {$\mathrm{RNA}_i$};
    \node (Pi) [below=3cm of R] {$\mathrm{P}_i$};
    \node (RP) [right=3cm of R] {$\mathrm{RNA}_i$--$\mathrm{P}_j$};
    \node (degrade) [below=3cm of RP] {Degradation};
    \node (Pj) [above right=2cm and 2cm of RP] {$\mathrm{P}_j$};
    
    \draw[produces] (D) to node[pos=0.5, swap] {$tr_i$} (R);
    \draw[converts] (R) to node[pos=0.75] {$d_{RNA}$} (degrade);
    \draw[converts] (RP) to node[pos=0.5] {$d_{i,j}$} (degrade);
    \draw[converts] (Pi) to node[pos=0.5, swap] {$d_{P_i}$} (degrade);
    
    \draw[produces] (R)  to node[pos=0.5, swap] {$tl_i$} (Pi);
    \draw[produces] (RP) to node[pos=0.75, swap] {$tl_{i,j}$} 
      (Pi);

    \draw[reversible] (R)  to node[pos=0.5] {$k_f$} (RP);
    \draw[reversible] (RP) to node[pos=0.5,swap] {$k_b$} (R);

    \draw[reversible] (Pj) to node[pos=0.5] {$k_f$} (RP);
    \draw[reversible] (RP) to node[pos=0.5] {$k_b$} (Pj);

  \end{tikzpicture}
  \caption{
  \textbf{Diagram of the general reaction network} 
  for the $i^{\mathrm{th}}$ protein in the network. 
  Dotted lines show production, where the molecule at the start
  of the arrow are not consumed; full lines show where molecules are converted
  to other types.
  \label{fig:simulation}}
\end{figure}

In order to capture PPR activity, we need to model the interaction between the
cell's expression machinery and RNA-PPR binding.
This is most simply done using and ODE model, where each process is modelled
using a rate equation.

The first two processes to model are transcription and translation. 
Assuming that the substrates necessary for RNA and protein production are
readily available and inexhaustible, these processes simple to
model as neither DNA nor RNA are consumed in either reaction.
This is not always the case in a cell, but for low enough rates of
transcription and translation this assumption holds.
Thus, the rate of production is simply proportional to the amount of DNA or RNA
present respectively.

We also need to model the production of the \ce{RNA_i-P_j} complex, as shown in
\ref{eqn:binding}, where \ce{P_j} represents the $j^{\mathrm{th}}$ protein in 
the model.
Note that this reaction is reversible, and so two rates are required -- one
representing the forwards reaction and one representing the backwards
direction.
\begin{equation}
  \cee{ RNA_i + P_j <=> RNA_i-P_j }
  \label{eqn:binding}
\end{equation}

In addition to these processes, we need to model translation of the
\ce{RNA_i-P_j} complex and also degradation of \ce{RNA}, \ce{P} and \ce{RNA-P}.
Each of these reactions is simple to model in isolation as the rate is simply
proportional to the concentration of the molecule, as we again assume that all
the necessary substrates are available.
Figure~\ref{fig:simulation} summarises these reactions for the $i^\mathrm{th}$
protein.

The differential equations governing the concentration of each molecule are 
shown in \ref{eqn:reaction_equations}.

\begin{align}
  \frac{d\cee{[RNA_i]}}{dt} &= tr_i 
    - \cee{[RNA_i]} \cdot d_{RNA_i} 
    - \sum_j \left(\cee{[RNA_i]}\cee{[P_j]} \cdot k_{f_{i,j}} 
      - \cee{[RNA_i-P_j]} \cdot k_{b_{i,j}} \right)  i
                                                  \nonumber \\
  \frac{d\cee{[P_i]}}{dt} &= 
    \cee{[RNA_i]}\cdot tl_i + \sum_j \cee{[RNA_j-P_i]}\cdot tl_{C_{i,j}} 
    - \cee{[P_i]}\cdot d_{P_i} \nonumber \\ 
    &\qquad {} - \sum_j \left(\cee{[RNA_i]}\cee{[P_j]} \cdot k_{f_{i,j}} 
      - \cee{[RNA_i-P_j]} \cdot k_{b_{i,j}} \right) 
                                                  \nonumber \\
  \frac{d\cee{[RNA_i-P_j]}}{dt} &=
      \left(\cee{[RNA_i]}\cee{[P_j]} \cdot k_{f_{i,j}} 
        - \cee{[RNA_i-P_j]} \cdot k_{b_{i,j}} \right)
      - \cee{[RNA_i-P_j]} \cdot d_{C_{i,j}}
                                                \label{eqn:reaction_equations}
\end{align}
%
Where:
%
\begin{align*}
  tr_i &= \text{is the translation rate of the $i^{\mathrm{th}}$ protein} \\
  d_{RNA_i} &= \text{degradation rate of \ce{RNA_i}} \\
  d_{C_{i,j}} &= \text{degradation rate of \ce{RNA_i-P_j} complex} \\
  k_{f_{i,j}}  &= \text{rate of association of \ce{RNA_i} and \ce{P_j}}\\
  k_{b_{i,j}} &= \text{rate of dissassociation of \ce{RNA_i} and \ce{P_j}}
\end{align*}

This system can be easily converted into a state-space formulation by defining
the state space to be the concatenation of the concentrations of each molecule
in the system.

These equations contain a large number of parameters for which numerical values
are needed in order to simulate a network of PPRs.
We can improve the situation by making some simplifying assumptions, namely
\begin{enumerate}
  \item \label{assm:transcription}
    Transcription rates are either high or low, depending on input conditions
  \item \label{assm:translation}
    Free \ce{RNA} transcripts are either highly translated or very slowly
    translated
  \item \label{assm:complex}
    \ce{RNA-P} complexes either have high translation rates and long half-lives
    (excitatory PPR binding) or low translation rates and short half-lives
    (repressive PPR binding)
\end{enumerate}

These three assumptions reduce the number of parameters required and suggest a
compact graphical representation of a network.
Each protein is represented by a circle which contains a symbol indicating
whether the protein is naturally translated or not.
Proteins which are produced in response to inputs to the system contain instead 
a letter indicating the input to which they respond.
PPR interactions are represented by lines between the proteins -- an arrow
between A and B represents that protein A binds to RNA B causing excitatory
binding, while a perpendicular bar represents repressive binding.

\begin{table}
  \centering
  \begin{tabular}{l | c | c }
    Description  & Value \\ \hline
    High transcription rate & $30 \mathrm{transcripts} \mathrm{minute}^{-1}$ \\
    Low transcription rate & $0.03 \mathrm{transcripts} \mathrm{minute}^{-1}$ \\
    On translation rate & $0.693 \mathrm{proteins}
      \mathrm{(transcript}\times \mathrm{minute)}^{-1}$ \\
    RNA half-life & $2$ minutes \\
    Protein half-life & $40$ minutes \\
    Long Complex half-life & 10 minutes \\
    Short Complex half-life & 1 minute \\
    Binding Rate & 0.5 $\mathrm{minute}^{-1}$ \\
    Unbinding Rate & $4\times 10^{-9}$ $\mathrm{minute}^{-1}$ \\
  \end{tabular}
  \caption{Rates used for the simulations shown in sections~\ref{sec:sim_logic}
  and~\ref{sec:sim_tabor}. Each was taken from estimates from the literature,
  although only the ration of binding to unbinding rate has been measured (via
  the disassociation constant. The plots shown were stable for a large range of
  binding rate.}
  \label{tab:sim_values}
\end{table}

\section{Simulation Results for Logic Gates}
\label{sec:sim_logic}

Figures \ref{fig:not_simulation}, \ref{fig:or_simulation} and
\ref{fig:nand_simulation} show simulations of NOT, OR and NAND logic gates
respectively using figures derived from the literature  
\citep[see][]{So2011,Andersen1998}, 
although the plots shown varied little over a large range of parameter values.

It is impossible to conclude with certainty before these simulations are
validated experimentally, but they do suggest that interesting and useful logic 
operations are indeed achievable using known PPR-mRNA interactions.

A major limitation of this model is that only one PPR may affect a transcript
at a time and there is no attempt to model competition between different PPRs 
over a binding site.
Both of these things can happen in reality since there is nothing to prevent a 
second PPR from binding at a different location on the same transcript.
We could attempt to model these interactions (possibly using stochastic
simulations rather than ODEs), but since so very little is known of the 
molecular processes driving PPR-RNA interactions such a model would be mostly
speculation.

\begin{figure}
  \begin{center}
    \begin{subfigure}{0.25\textwidth}
      \centering
      \begin{tikzpicture}
        \node[ppr] (p1) {$A$};
        \node[ppr, right of=p1] (p2) {$+$};
        \node[right of=p2] (out) {out};

        \draw[repress] (p1) to (p2);
        \draw[->] (p2) to (out);
      \end{tikzpicture}
    \end{subfigure}
    ~
    \begin{subfigure}{0.7\textwidth}
      \centering
      \begin{tikzpicture}
        \begin{axis}[
            xlabel={time,min},
            ylabel={concentration},
            height=5cm,
            width=1.0\textwidth,
            cycle list name=color list,
            legend pos=north west,
            ytick = \empty,
            ticklabel style={ 
              /pgf/number format/fixed, 
              /pgf/number format/precision=5 
            }, scaled ticks=false 
          ]
          \addplot table[x=time, y=0]{Data/logic_NOT.dat};
          \addplot table[x=time, y=1]{Data/logic_NOT.dat};
          \legend{0,1};
        \end{axis}
      \end{tikzpicture}
    \end{subfigure}
  \end{center}
  \caption{
    \textbf{An implementation of a NOT gate.}
    Expression of $A$ causes the output to be repressed.
    \label{fig:not_simulation}}

  \begin{center}
    \begin{subfigure}{0.25\textwidth}
      \centering
      \begin{tikzpicture}
        \node[ppr] (p1) {$A$};
        \node[ppr, below right=0.5cm and 1cm of p1] (p3) {$-$};
        \node[ppr, below left=0.5cm and 1cm of p3] (p2) {$B$};
        \node[right of=p3] (out) {out};

        \draw[induce, bend left] (p1) to (p3);
        \draw[induce, bend right] (p2) to (p3);
        \draw[->] (p3) to (out);
      \end{tikzpicture}
    \end{subfigure}
    ~
    \begin{subfigure}{0.7\textwidth}
      \centering
      \begin{tikzpicture}
        \begin{axis}[
            xlabel={time,min},
            ylabel={concentration},
            height=5cm,
            width=1.0\textwidth,
            cycle list name=color,
            no markers,
            legend pos=north west,
            ytick = \empty,
            ticklabel style={ 
              /pgf/number format/fixed, 
              /pgf/number format/precision=5 
            }, scaled ticks=false 
          ]
          \addplot table[x=time, y=00]{Data/logic_OR.dat};
          \addplot table[x=time, y=01]{Data/logic_OR.dat};
          \addplot table[x=time, y=10]{Data/logic_OR.dat};
          \addplot table[x=time, y=11]{Data/logic_OR.dat};
          \legend{00,01,10,11};
        \end{axis}
      \end{tikzpicture}
    \end{subfigure}
  \end{center}
  \caption{\textbf{An implementation of an OR gate.}
    Expression of either $A$ or $B$ causes excitation of the output.
    This can be easily expanded to $N$ inputs using a total of $N$ (possibly
    identical) PPRs.
    \label{fig:or_simulation}}

  \begin{center}
    \begin{subfigure}{0.25\textwidth}
      \centering
      \begin{tikzpicture}[node distance=0.8cm]
        \node[ppr] (p1) {$A$};
        \node[ppr, right of=p1] (p3) {+};
        \node[ppr, below right=0.5cm and 0.8cm of p3] (p5) {$-$};
        \node[ppr, below left =0.5cm and 0.8cm of p5] (p4) {+};
        \node[ppr, left of=p4] (p2) {$B$};
        \node[right of=p5] (out) {out};

        \draw[repress] (p1) to (p3);
        \draw[repress] (p2) to (p4);
        \draw[induce, bend left] (p3) to (p5);
        \draw[induce, bend right] (p4) to (p5);
        \draw[->] (p5) to (out);
      \end{tikzpicture}
    \end{subfigure}
    ~
    \begin{subfigure}{0.7\textwidth}
      \centering
      \begin{tikzpicture}
        \begin{axis}[
            xlabel={time,min},
            ylabel={concentration},
            height=5cm,
            width=1.0\textwidth,
            cycle list name=color,
            no markers,
            legend pos=north west,
            ytick = \empty,
            ticklabel style={ 
              /pgf/number format/fixed, 
              /pgf/number format/precision=5 
            }, scaled ticks=false 
          ]
          \addplot table[x=time, y=00]{Data/logic_NAND.dat};
          \addplot table[x=time, y=01]{Data/logic_NAND.dat};
          \addplot table[x=time, y=10]{Data/logic_NAND.dat};
          \addplot table[x=time, y=11]{Data/logic_NAND.dat};
          \legend{00,01,10,11};
        \end{axis}
      \end{tikzpicture}
    \end{subfigure}
  \end{center}
  \caption{\textbf{An implementation of a NAND gate.}
    Output is produced unless both $A$ and $B$ are high.
    This can be implemented for $N$ inputs with $2N$ PPRs.
    \label{fig:nand_simulation}}
\end{figure}

\section{Implementing Tabor's Edge Detector}
\label{sec:sim_tabor}

We can use this model to simulate an implementation of Tabor's bacterial edge 
detector (introduced in section \ref{sec:synbio}).
The system described in \citet{edgeDetector} produces pigment in response to
the presence of AHL and light.
Our inputs to the PPR network are the AHL signal ($S$) and dark
($\overline{L}$), and output a pigment signal ($P$) according to 
\begin{align*}
  P &= S \cdot L \\
  P &= \overline{\overline{S} + \overline{L}}
\end{align*}
using De Morgan's law.

Figure \ref{fig:tabor_simulation} shows an implementation of this logic using
simulated PPR proteins.
The key advantage of this solution is that knowledge of a particular promoter
which responds to the chosen cell-cell signal molecule used the output of the
light detector is not required, we merely need to be able to detect the two
signals independently.
This would allow us to freely choose another cell-cell signalling molecule
(perhaps with a different diffusion rate) in order to change the behaviour of
the system -- all we require is a promoter which will respond to it.

\begin{figure}
  \begin{center}
    \begin{subfigure}{0.25\textwidth}
      \centering
      \begin{tikzpicture}[node distance=0.8cm]
        \node[ppr] (p0) {$\overline{L}$};
        \node[ppr, below right=0.5cm and 0.8cm of p0] (p3) {$-$};
        \node[ppr, below left =0.5cm and 0.8cm of p3] (p2) {$+$};
        \node[ppr, left of=p2] (p1) {$S$};
        \node[ppr, right of=p3] (p4) {$+$};
        \node[right of=p4] (out) {$P$};

        \draw[repress] (p1) to (p2);
        \draw[repress] (p3) to (p4);
        \draw[induce, bend left] (p0) to (p3);
        \draw[induce, bend right] (p2) to (p3);
        \draw[->] (p4) to (out);
      \end{tikzpicture}
    \end{subfigure}
    ~
    \begin{subfigure}{0.7\textwidth}
      \centering
      \begin{tikzpicture}
        \begin{axis}[
            xlabel={time,min},
            ylabel={concentration},
            height=5cm,
            width=1.0\textwidth,
            cycle list name=color,
            no markers,
            legend pos=north west,
            ytick = \empty,
          ]
          \addplot table[x=time, y=00]{Data/logic_Tabor.dat};
          \addplot table[x=time, y=01]{Data/logic_Tabor.dat};
          \addplot table[x=time, y=10]{Data/logic_Tabor.dat};
          \addplot table[x=time, y=11]{Data/logic_Tabor.dat};
          \legend{00,01,10,11};
        \end{axis}
      \end{tikzpicture}
    \end{subfigure}
  \end{center}
  \caption{\textbf{An implementation of Tabor's bacterial edge detector.}
    Pigment is only produced in response to light ($\overline{L}=0$) and signal
    ($S=1$).
    \label{fig:tabor_simulation}}
\end{figure}

