
\chapter{Simulating PPR Activity} 
\label{chap:simulation}

Interesting to simulate networks of PPRs to demonstrate logic

\section{Simulation Model}
\label{sec:sim_model}

A discussion of the ODE model used for simulation and where the parameters came
from

\begin{figure}
  \centering
  \begin{tikzpicture}[auto,
      converts/.style={->, >=triangle 60},
      produces/.style={->, >=triangle 60, dashed},
    reversible/.style={->, >=triangle 60, bend left=20}]
    \node (D) {$\mathrm{DNA}_i$};
    \node (R) [below=2cm of D] {$\mathrm{RNA}_i$};
    \node (Pi) [below=3cm of R] {$\mathrm{P}_i$};
    \node (RP) [right=3cm of R] {$\mathrm{RNA}_i$--$\mathrm{P}_j$};
    \node (degrade) [below=3cm of RP] {Degradation};
    \node (Pj) [above right=2cm and 2cm of RP] {$\mathrm{P}_j$};
    
    \draw[produces] (D) to node[pos=0.5, swap] {$tr_i$} (R);
    \draw[converts] (R) to node[pos=0.75] {$d_{RNA}$} (degrade);
    \draw[converts] (RP) to node[pos=0.5] {$d_{i,j}$} (degrade);
    \draw[converts] (Pi) to node[pos=0.5, swap] {$d_{P_i}$} (degrade);
    
    \draw[produces] (R)  to node[pos=0.5, swap] {$tl_i$} (Pi);
    \draw[produces] (RP) to node[pos=0.75, swap] {$tl_{i,j}$} 
      (Pi);

    \draw[reversible] (R)  to node[pos=0.5] {$k_f$} (RP);
    \draw[reversible] (RP) to node[pos=0.5,swap] {$k_b$} (R);

    \draw[reversible] (Pj) to node[pos=0.5] {$k_f$} (RP);
    \draw[reversible] (RP) to node[pos=0.5] {$k_b$} (Pj);

  \end{tikzpicture}
  \caption{    \label{fig:simulation}}
\end{figure}

\section{Simulation Results for Logic Gates}
\label{sec:sim_logic}

\begin{figure}
  \begin{center}
    \begin{subfigure}{0.25\textwidth}
      \centering
      \begin{tikzpicture}
        \node (A) {A};
        \node[ppr, right of=A] (p1) {};
        \node[ppr, right of=p1] (p2) {+};
        \node[right of=p2] (out) {out};

        \draw[->] (A) to (p1);
        \draw[repress] (p1) to (p2);
        \draw[->] (p2) to (out);
      \end{tikzpicture}
    \end{subfigure}
    ~
    \begin{subfigure}{0.7\textwidth}
      \centering
      \begin{tikzpicture}
        \begin{axis}[
            xlabel={time,min},
            ylabel={concentration},
            height=5cm,
            width=1.0\textwidth,
            cycle list name=color list,
            legend pos=north west,
          ]
          \addplot table[x=time, y=0]{Data/logic_NOT.dat};
          \addplot table[x=time, y=1]{Data/logic_NOT.dat};
          \legend{0,1};
        \end{axis}
      \end{tikzpicture}
    \end{subfigure}
  \end{center}
  \caption{CAPTION HERE
    \label{fig:not_simulation}}

  \begin{center}
    \begin{subfigure}{0.25\textwidth}
      \centering
      \begin{tikzpicture}
        \node (A) {A};
        \node[below=1cm of A] (B) {B};
        \node[ppr, right of=A] (p1) {};
        \node[ppr, right of=B] (p2) {};
        \node[ppr, below right=0.5cm and 1cm of p1] (p3) {$-$};
        \node[right of=p3] (out) {out};

        \draw[->] (A) to (p1);
        \draw[->] (B) to (p2);
        \draw[induce, bend left] (p1) to (p3);
        \draw[induce, bend right] (p2) to (p3);
        \draw[->] (p3) to (out);
      \end{tikzpicture}
    \end{subfigure}
    ~
    \begin{subfigure}{0.7\textwidth}
      \centering
      \begin{tikzpicture}
        \begin{axis}[
            xlabel={time,min},
            ylabel={concentration},
            height=5cm,
            width=1.0\textwidth,
            cycle list name=color,
            no markers,
            legend pos=north west,
          ]
          \addplot table[x=time, y=00]{Data/logic_OR.dat};
          \addplot table[x=time, y=01]{Data/logic_OR.dat};
          \addplot table[x=time, y=10]{Data/logic_OR.dat};
          \addplot table[x=time, y=11]{Data/logic_OR.dat};
          \legend{00,01,10,11};
        \end{axis}
      \end{tikzpicture}
    \end{subfigure}
  \end{center}
  \caption{CAPTION HERE
    \label{fig:or_simulation}}

  \begin{center}
    \begin{subfigure}{0.25\textwidth}
      \centering
      \begin{tikzpicture}[node distance=0.8cm]
        \node (A) {A};
        \node[below=1cm of A] (B) {B};
        \node[ppr, right of=A] (p1) {};
        \node[ppr, right of=B] (p2) {};
        \node[ppr, right of=p1] (p3) {+};
        \node[ppr, right of=p2] (p4) {+};
        \node[ppr, below right=0.5cm and 0.8cm of p3] (p5) {$-$};
        \node[right of=p5] (out) {out};

        \draw[->] (A) to (p1);
        \draw[->] (B) to (p2);
        \draw[repress] (p1) to (p3);
        \draw[repress] (p2) to (p4);
        \draw[induce, bend left] (p3) to (p5);
        \draw[induce, bend right] (p4) to (p5);
        \draw[->] (p5) to (out);
      \end{tikzpicture}
    \end{subfigure}
    ~
    \begin{subfigure}{0.7\textwidth}
      \centering
      \begin{tikzpicture}
        \begin{axis}[
            xlabel={time,min},
            ylabel={concentration},
            height=5cm,
            width=1.0\textwidth,
            cycle list name=color,
            no markers,
            legend pos=north west,
          ]
          \addplot table[x=time, y=00]{Data/logic_NAND.dat};
          \addplot table[x=time, y=01]{Data/logic_NAND.dat};
          \addplot table[x=time, y=10]{Data/logic_NAND.dat};
          \addplot table[x=time, y=11]{Data/logic_NAND.dat};
          \legend{00,01,10,11};
        \end{axis}
      \end{tikzpicture}
    \end{subfigure}
  \end{center}
  \caption{CAPTION HERE
    \label{fig:nand_simulation}}
\end{figure}

Demonstrations of various simple logic gates

\section{Implementing Tabor's Edge Detector}
\label{sec:sim_tabor}

Discussion of an implementation of Tabor's bacterial edge detector using PPRs

