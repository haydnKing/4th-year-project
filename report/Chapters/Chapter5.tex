
\chapter{Conclusions and Further Work}
\label{chap:Conclusions} 

\section{Summary of the Work}

Pentatricopeptide repeat proteins are a vital class of proteins for a large
number of plants.
During this project, software was developed to automatically identify, extract 
and characterise PPRs from un-annotated DNA sequence.

A comparison of current theories regarding the prediction of PPR binding
preferences was made, and it was shown that while current models of PPR binding
sites score a genuine match highly, they allow too much variation in the
sequence such that any reasonably long random sequence will contain matches 
which are just as good if not better, making accurate prediction impossibly.

It was also shown that there appears to be considerable conservation of PPR 
binding sites across different organisms.
If this is indeed the case then there must also be considerable conservation of
both PPR function and PPR binding preferences across the same organisms.
This conservation would make experimentally extracting and characterising new
proteins from these organisms relatively simple, drastically increasing the
number of PPR-RNA pair of which we are aware.

In order to assess the potential of the PPR protein as a tool for the wider
field of synbio, a general model of PPR activity based on known interactions
was created.
By simulating various interaction networks, it was shown that PPRs can be 
used to perform various simple logic operations at the translation level within
the cell.
Since each PPR can be programmed to recognise a large sequence of RNA (up to at
least $30$bp) this leads to an enormous address space for signalling (a maximum
of $4^{30}$), solving the problem of cross-talk in molecular circuit design.

\section{Future Work and Directions}

Discussion of the limits of our knowledge of PPRs -- limited information about
binding rules, details of the molecular interactions causing increased or
reduced translation and degradation, or of the mechanism or true purpose of RNA
editing.

