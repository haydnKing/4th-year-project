
\chapter{Conclusions and Further Work}
\label{chap:Conclusions} 

\lettrine{A}{} brief summary of the work carried out during the project and a 
description of the work required to make PPR technology a reality are given in 
this chapter.

\section{Summary of the Work}

Pentatricopeptide repeat proteins are a vital class of proteins for a large
number of plants.
During this project, software was developed to automatically identify, extract 
and characterise PPRs from unannotated DNA sequence.
It was found that the majority of plants for which nuclear genomes are 
available contain roughly the same number of PPR proteins, suggesting that the
function of these proteins is consistent across these plants.

A comparison of current theories regarding the prediction of PPR binding
preferences was made, and it was shown that while current models of PPR binding
sites score a genuine match highly, they allow too much variation in the
sequence such that any reasonably long random sequence will contain matches 
which are just as good if not better, making accurate prediction impossible.

It was also shown that there appears to be considerable conservation of PPR 
binding sites across different organisms.
If this is confirmed experimentally then there must also be considerable 
conservation of both PPR function and PPR binding preferences across the same 
organisms.
This conservation would make experimentally extracting and characterising new
proteins from these organisms relatively simple, which would drastically 
increase the number of PPR-RNA pairs of which we are aware.

In order to assess the potential of the PPR protein as a tool for the wider
field of synthetic biology, a general model of PPR activity based on known 
interactions was created.
By simulating various interaction networks, it was shown that PPRs can be 
used to perform various simple logic operations at the translation level within
the cell.
Since each PPR can be programmed to recognise a large sequence of RNA (up to at
least $30$bp) this leads to an enormous address space for signalling (a maximum
of $4^{30}$), effectively solving the problem of cross-talk in molecular 
circuit design.

\section{Future Work and Directions}

A thorough understanding of the rules governing binding domain preference in
only part of the way to realising the potential of PPR-based technology.
An understanding of the molecular basis behind it's interactions with
translation of mRNA transcripts is also vital in order to exploit these
interactions for our own purposes.

Plausible theories exist suggesting RNA stabilisation, ribosome recruitment and
degradation of mRNA, and there is some evidence for some of these interactions
\emph{in vitro} \citep[see][]{Prikryl2011}, but there is little hard evidence 
that these mechanisms occurs naturally \emph{in vivo}.
Being able to target PPRs to specific regions of the mRNA message will
certainly aid with this investigation, as will knowledge of the three
dimensional structure of the PPR-RNA complex.

